\section{Faza implementacji}
W tym rozdziale opisano część implementacji aplikacji. Rozwiązania i koncepty dotyczą systemu operacyjnego Android. Fragmenty kodu zostały uproszczone na potrzeby tego rozdziału. Między innymi, dodano komentarze w języku polskim oraz pominięto część implementacji. W załącznikach dostępny jest szczegółowy diagram klas, z wyłączeniem niektórych implementacji.

\subsection{Architektura aplikacji}
Architektura aplikacji została zaprojektowana w oparciu o model z jedną główną aktywnością która pełni rolę centralnego punktu zarządzania nawigacją oraz cyklem życia aplikacji. W aplikacji istnieje także aktywność dedykowana do obsługi płatności, co pozwala na oddzielenie logiki związanej z transakcjami od głównych funkcji aplikacji. Za poszczególne ekrany interfejsu użytkownika odpowiadają fragmenty.

\subsection{Zarządzanie stanem aplikacji}
Aplikacja implementuje klasę \tcbox{Application} z przeznaczeniem na przechowywanie stanu aplikacji oraz zarządzaniem procesem inicjalizacji po uruchomieniu. Jest to specjalna klasa której instancję system tworzy przed jakąkolwiek inną klasą w aplikacji. Klasa ta ma cykl życia niezależny od cyklu życia aktywności, trwający przez cały czas pracy aplikacji. Dzięki temu stan aplikacji jest niewrażliwy na wszelkie zmiany w konfiguracji które powodują ponowne uruchomienie aktywności.

\newpage

\subsection{Foreground service}
\tcbox{Service} to komponent aplikacji, który potrafi wykonywać długotrwałe operacje w tle nawet po tym gdy użytkownik przejdzie do innej aplikacji. \tcbox{Foreground service} jest to specjalny rodzaj usługi działającej na pierwszym planie wykonującej operacje zauważalne dla użytkownika. Usługa tego typu ma obowiązek wyświetlać stałe powiadomienie aby informować użytkownika o procesie działającym w tle. W nowszych wersjach Androida powiadomienie to może być odrzucone przez użytkownika. Przykładem takiej usługi jest odtwarzacz multimedialny z interfejsem w formie powiadomienia wyświetlanego gdy jakaś aplikacja odtwarza w tle muzykę.\\

Połączenie odpowiedniej konfiguracji z użyciem \tcbox{Foreground service} zmniejsza szanse na zabicie aplikacji przez system. Tym sposobem aplikacja może działać w tle po tym gdy użytkownik ją zamknie. Jest to kluczowa funkcjonalność z racji tego że MQTT jest protokołem komunikacyjnym w czasie rzeczywistym.\\

Niestety praca w tle aplikacji jest obłożona szeregiem obostrzeń w imię wydłużenia czasu pracy na baterii. Dla przykładu jeżeli aplikacja obsługuje Android w wersji 14 wzwyż, deweloper jest zobowiązany do zadeklarowania typu \tcbox{Foreground service}. (dokumentacja - \cite{fgstyperequired}) Pula tych typów jest ograniczona, a każdych z nich ma swoje ściśle unormowane zastosowanie. Istnieje również typ zarezerwowany dla specjalnych przypadków których pozostałe typy nie pokrywają. Wybranie tego typu wiąże się z obowiązkiem podania szczegółowego wytłumaczenia, jak i również przygotowania nagrania tego w jaki sposób nasza aplikacja wykorzystuje usługę oznaczoną tym typem. Mimo spełnienia tych wymogów w dalszym ciągu możemy spotkać się z odmową przy próbie publikacji aplikacji na platformie Google Play.

\newpage

\subsection{Daemons}
Aplikacja w początkowych założeniach miała obsługiwać więcej niż jeden protokół komunikacyjny. Rozwiązania wprowadzone na podstawie tego wymagania stanowią fundamentalną część implementacji. W związku z powyższym, mimo tego że wymaganie to zostało jednak na ten moment porzucone, dalej ma ono wpływ na obecną architekturę aplikacji.

\subsubsection{Daemon}
Demon (ang. daemon) to termin używany w informatyce, który odnosi się do programu działającego w tle, zazwyczaj bez interakcji z użytkownikiem. Może on wykonywać różne zadania, takie jak monitorowanie systemu, zarządzanie zasobami czy obsługa żądań sieciowych.\\

W kontekście przedstawianej aplikacji jest to abstrakcyjna klasa nadrzędna po której dziedziczą wszystkie inne demony. Pełnią one rolę źródła informacji i zdarzeń. Każdy dashboard ma przypisany dedykowany demon danego rodzaju, zależnie od tego z wykorzystaniem jakiego protokołu się on komunikuje. Rolą tego konceptu jest odseparowanie części architektury odpowiedzialnej za komunikację.\\

Poniżej opisano niektóre części klasy \tcbox{Daemon}.

\begin{lstlisting}[language=Kotlin]
//Czy demon został zwolniony z użytku
//Demon zostaje zwolniony wtedy gdy nie jest już dłużej używany
//Na przykład gdy dashboard do którego jest przypisany zostaje usunięty
var isDischarged = false

//Czy demon jest włączony/czy pracuje
protected abstract val isEnabled: Boolean

//Specjalne obserwowalne pole
//Jej wartość zmienia się w momencie gdy zmienia się stan demona
//Poszczególne części aplikacji mogą nasłuchiwać nowych wartości
//na tym polu w związku z czym być powiadamiane o zmianie stanu demona
abstract val statePing: MutableLiveData<String?>

//Stan demona
//Konkretna implementacja tego pola jest zależna od rodzaju demona
abstract val state: Any
\end{lstlisting}

\newpage

\begin{lstlisting}[language=Kotlin]
//Metoda uruchamiana w momencie gdy demon jest przydzielany do dashboardu
open fun notifyAssigned() {
    isDischarged = false
}

//Metoda uruchamiana w momencie gdy demon jest zwalniany z użytku
open fun notifyDischarged() {
    isDischarged = true
}

//Metoda uruchamiana w momencie zmiany konfiguracji demona
//np. zmiana adresu serwera, portu, itp.
open fun notifyConfigChanged() {}
\end{lstlisting}

\vspace{1em}

Dodatkowo klasa \tcbox{Daemon} posiada \tcbox{companion object} definiujący \tcbox{invoke operator}.\\

W Kotlinie \tcbox{companion object} jest to miejsce gdzie zdefiniowane są metody i właściwości które są dostępne bez tworzenia instancji klasy.\\
Natomiast \tcbox{invoke operator} pozwala na wywołanie obiektu jako funkcji.\\

\begin{lstlisting}[language=Kotlin]
companion object {
    operator fun invoke(context: Context, dashboard: Dashboard, type: Type) =
    when (type) {
        Type.MQTTD -> Mqttd::class
        Type.BLUETOOTHD -> Bluetoothd::class
    }.constructors.first().call(context, dashboard)
}
\end{lstlisting}

Użycie w kodzie \tcbox{Daemon()} zamiast "tworzyć instancję"\ abstrakcyjnej klasy \tcbox{Daemon} uruchamia funkcję \tcbox{invoke} która zwraca instancję klasy danego typu demona.

\subsubsection{Daemonized}
Interfejs który implementuje jedynie klasa \tcbox{Tile}. Powstał on w celu odseparowania tej części kafelki dla przejrzystości.\\

Zawiera on już zaimplementowane metody \tcbox{send} oraz \tcbox{receive}. Metoda \tcbox{send} służy do wysyłania nowych wiadomości. Natomiast metoda \tcbox{receive} używana jest do przekazania przychodzącej wiadomości do kafelki.\\

Dodatkowo istnieją metody \tcbox{onSend} oraz \tcbox{onReceive} wykorzystywane przez te wyżej wspomniane. Obie z przeznaczeniem bycia zaimplementowanym przez docelowy element.

\newpage

\subsubsection{Mqttd}
Klasa pełniąca rolę demona dla dashboardów komunikujących się z wykorzystaniem protokołu MQTT. Zawiera następujące pola:

\begin{lstlisting}[language=Kotlin]
//Asynchroniczny klient MQTT
private var client: Mqtt5AsyncClient? = null

//Obecna konfiguracja
private var currentConfig = MqttConfig()

//Lista topic która obecnie subskrybuje client
private var topics: MutableList<Pair<String, Int>> = mutableListOf()

//Pola dziedziczone po klasie Deamon

public override val isEnabled
    get() = d.mqtt.isEnabled && !isDischarged

private val isConnected
    get() = client?.config?.state?.isConnected ?: false

override val statePing: MutableLiveData<String?> = MutableLiveData(null)

override val state: State
    get() = if (manager.isWorking) State.ATTEMPTING
    else try {
        if (!isConnected) State.DISCONNECTED
        else if (currentConfig.sslRequired && !currentConfig.sslTrustAll) {
            State.CONNECTED_SSL
        } else State.CONNECTED
    } catch (e: Exception) {
        Debug.recordException(e)
        State.FAILED
    }
\end{lstlisting}

\newpage

\tcbox{Mqttd} korzysta z \tcbox{StatusManager}. Jego zasada działania opisana jest w dalszym rozdziale.

\begin{lstlisting}[language=Kotlin]
inner class Manager : StatusManager() {
    override fun isStable(): Boolean {
        return isConnected == isEnabled && (currentConfig == d.mqtt || !isEnabled)
    }

    override fun makeStable() {
        if (isEnabled) {
            if (isConnected) {
                Debug.log("MQTT_DISCONNECT")
                client?.disconnect()
            } else {
                if (currentConfig != d.mqtt || client == null) {
                    Debug.log("MQTT_BUILD_NEW_CLIENT")
                    buildClient(d.mqtt.copy())
                }
                Debug.log("MQTT_CONNECT")
                client?.connect()
            }
        } else {
            Debug.log("MQTT_CLOSE")
            client?.disconnect()
        }
    }

    override fun onJobDone() = statePing.postValue("")
    override fun onJobStart() = statePing.postValue(null)
    override fun onException(e: Exception) {
        super.onException(e)
        when (e) {
            is IllegalArgumentException -> statePing.postValue(e.message)
        }
    }
}
\end{lstlisting}

\newpage

Kluczową częścią klasy \tcbox{Mqttd} jest funkcja \tcbox{buildClient}. Bierze ona \tcbox{MqttConfig} jako parametr i na jego podstawie buduje odpowiedniego klienta.

\begin{lstlisting}[language=Kotlin]
fun buildClient(config: MqttConfig) {
    var client = Mqtt5Client.builder()
        .identifier(config.clientId)
        .serverHost(config.address)
        .serverPort(config.port)
        .addConnectedListener {
            Debug.log("MQTT_ON_CONNECTED")
            topicCheck()
            statePing.postValue("")
        }
        .addDisconnectedListener {
            Debug.log("MQTT_ON_DISCONNECTED")
            topics = mutableListOf()
            manager.dispatch(reason = "connection")

            if (it.cause !is ConnectionFailedException) statePing.postValue("")
            else statePing.postValue(it.cause.cause?.message)
        }

    //W razie potrzeby konfiguruje autoryzację
    if (config.includeCred) client = client.simpleAuth()
        .username(config.username)
        .password(config.pass.toByteArray())
        .applySimpleAuth()

        //W razie potrzeby konfiguruje użycie protokołu WebSocket
        if (config.protocol == Protocol.WS || config.protocol == Protocol.WSS) {
            client = client
                .webSocketConfig()
                .queryString(config.queryString)
                .serverPath(config.serverPath)
                .applyWebSocketConfig()
        }

    //W razie potrzeby konfiguruje szyfrowanie
    //Metoda setupSSL jest opisana niżej
    if (config.sslRequired) client = client.setupSSL(config)

    //Buduje klienta i aktualizuje obecny config
    this.client = client.buildAsync()
    currentConfig = config
}
\end{lstlisting}

\newpage

Poniżej znajduje się metoda odpowiedzialna za konfigurację:
\begin{itemize}[leftmargin=*]
    \item szyfrowania połaczenia oraz użycia certyfikatów i kluczy klienta
    \item związaną z wykorzystaniem niestandardowego urzędu certyfikacji (CA)
    \item związaną z opcją pominięcia weryfikacji certyfikatu serwera
\end{itemize}

\begin{lstlisting}[language=Kotlin]
private fun Mqtt5ClientBuilder.setupSSL(config: MqttConfig): Mqtt5ClientBuilder {
    //---------------------------------------------------------------------------
    //Tworzy magazyn kluczy z certyfikatem i kluczem klienta
    val kmfStore = KeyStore.getInstance(KeyStore.getDefaultType())
    val kmfKeyPassword = UUID.randomUUID().toString().toCharArray()
    kmfStore.load(null, null)
    kmfStore.setCertificateEntry("cc", config.clientCert)
    config.clientKey?.let {
        kmfStore.setKeyEntry(
            "k",
            it.private,
            kmfKeyPassword,
            arrayOf<Certificate?>(config.clientCert)
        )
    }
    //---------------------------------------------------------------------------
    //Tworzy magazyn kluczy używany do weryfikacji serwera
    //Opcjonalne dodawany jest certyfikat niestandardowego urzędu certyfikacji
    //Ma to na celu dodanie wsparcia dla self-signed certyfikatów serwera
    //zachowując jednocześnie bezpieczeństwo przed atakami typu MITM
    var tmfStore: KeyStore? = null
    if (config.caCert != null) {
        tmfStore = KeyStore.getInstance(KeyStore.getDefaultType())
        tmfStore.load(null, null)
        tmfStore.setCertificateEntry("c", config.caCert)
    }

    //KeyManager decyduje które dane uwierzytelniające
    //powinny zostać wysłane do hosta zdalnego
    val kmf = KeyManagerFactory.getInstance(KeyManagerFactory.getDefaultAlgorithm())
    kmf.init(kmfStore, kmfKeyPassword)

    val alg = TrustManagerFactory.getDefaultAlgorithm()

    //TrustManager określa czy zdalne połączenie powinno być uznawane za zaufane
    val tmf = if (!config.sslTrustAll) TrustManagerFactory.getInstance(alg)
    else TrustAllTrustManagerFactory()

    if (!config.sslTrustAll) tmf.init(tmfStore)

    return this.sslConfig()
        .keyManagerFactory(kmf)
        .trustManagerFactory(tmf)
        .applySslConfig()
}
\end{lstlisting}

\newpage

Metody związane z wiadomościami:

\begin{lstlisting}[language=Kotlin]
//Metoda używana do publikacji wiadomości
fun publish(topic: String, msg: String, qos: Int = 0, retain: Boolean = false)

//Metody służące do zarządzania subskrypcjami
private fun subscribe(topic: String, qos: Int)
private fun unsubscribe(topic: String, qos: Int)

//Buduje listę topic które powinny być subskrybowane
//Porównuje tą listę z obecnie subskrybowanymi topic
//W przypadku różnic korzysta z metod subscribe i/lub unsubscribe
private fun topicCheck()
\end{lstlisting}

\newpage

\subsubsection{MqttConfig}
Klasa zawierająca konfigurację dashboardu dla \tcbox{Mqttd}.\\
Zawiera następujące pola:

\begin{lstlisting}[language=Kotlin]
var isEnabled: Boolean
var sslTrustAll: Boolean
@JsonIgnore
var caCert: X509Certificate?
var caFileName: String
@JsonIgnore
var clientCert: X509Certificate?
var clientFileName: String
@JsonIgnore
var clientKey: KeyPair?
var keyFileName: String
var address: String
var protocol: Mqttd.Protocol
var port: Int
var keepAlive: Int
var includeCred: Boolean
var queryString: String
var serverPath: String
var username: String
var pass: String
var clientId: String

//Pole z getterem
//Wartość jest generowana na podstawie pola protocol
val sslRequired: Boolean

//Pola z setterem
//Przypisanie wartości powoduje wygenerowanie certyfikatów 
var caCertStr: String?
var clientCertStr: String?
var clientKeyStr: String?
\end{lstlisting}

Wartość części pól jest generowana na postawie wartości innego pola. Na przykład wartość pola \tcbox{caCert} jest generowana na podstawie wartości pola \tcbox{caCertStr}. Z tego powodu część pól jest oznaczona anotacją \tcbox{@JsonIgnore}.\\
Ma ona za zadanie wyłączyć te pola z procesu serializacji.

\newpage

\subsubsection{MqttDaemonizedConfig}
Klasa zawierająca konfigurację danej kafelki dla \tcbox{Mqttd}.\\
Zawiera następujące pola:

\begin{lstlisting}[language=Kotlin]
var isEnabled: Boolean
var lastReceive: Date?

//Mapa topic używanych przez kafelkę
//Kluczami są funkcje w obrębie kafelki
//Większość kafelek posiada jedną funkcję (np. button) więc i jedną parę topic
//Część jest bardziej skomplikowana (np. lights) i posiada wiecej takich par
val subs: MutableMap<String, String>
val pubs: MutableMap<String, String>

//Kafelki posiadają możliwość parsowania payload jako json
//To pole zawiera mapę ścieżek json dla każdej funkcji w obrębie kafelki
val jsonPaths: MutableMap<String, String>

//Pole zawierające payload jaki ma być wysłany w zależności od funkcji kafelki
var payloads: MutableMap<String, String>
var qos: Int

//Część kafelek ma możliwość dynamicznego ustawienia payload przed wysłaniem
//To pole zawiera informację czy payload jest statyczny
var payloadIsVar: Boolean

//Czy payload ma być parsowany jako json
var payloadIsJson: Boolean

//Czy wyświetlić komunikat z prośbą o potwierdzenie przed wysłaniem
var doConfirmPub: Boolean

var doRetain: Boolean
var doLog: Boolean
var doNotify: Boolean
var silentNotify: Boolean
var notifyTitle: String
var notifyPayload: String
\end{lstlisting}

\newpage

\subsection{Klasy}
\subsubsection{Dashboard}
Kluczowa klasa całej implementacji. Jak nazwa wskazuje ma reprezentować dashboard. Dziedziczy po klasie \tcbox{RecyclerViewItem}. Dla uproszczenia dziedziczone pola i metody zostały wyłączone z opisu poniżej.\\
Więcej na ten temat znajduje się w dziale \textbf{Recycler View}.\\

\begin{lstlisting}[language=Kotlin]
var name: String = ""

//Typ dashboardu
var type: Daemon.Type = Daemon.Type.MQTTD

//Klucz ikony dashboardu
var iconKey = "il_interface_plus_circle"

//Ikona dashboardu
//Wartość tego pola jest generowana przy każdym użyciu na podstawie pola iconKey
val iconRes: Int

//Kolor ikony dashboardu
var hsv: FloatArray

//Paleta kolorów ikony dashboardu
//Wartość tego pola jest generowana przy każdym użyciu na podstawie pola hsv
//Więcej na ten temat znajduje się w dziale Theme
val pallet: Theme.ColorPallet

//Dziennik logów dashboardu
var log = Log()

//Konfiguracja dla Mqttd
var mqtt = MqttConfig()

@JsonIgnore
//Demon przypisany do tego dashboardu
//Proces przypisywania demonów opisany jest w działach DaemonsManager oraz Setup
var daemon: Daemon? = null

//Lista kafelek jakie posiada dashboard
var tiles: MutableList<Tile>
\end{lstlisting}

\newpage

\subsubsection{Tile}
Bazowa klasa nadrzędna dla wszystkich kafelek. Po tej klasie dziedziczą pozostałe klasy reprezentujące już konkretny rodzaj kafelki (np. \tcbox{ButtonTile}). Podobnie jak dashboard dziedziczy ona po klasie \tcbox{RecyclerViewItem} oraz oczywiście implementuje interfejs \tcbox{Daemonized}.
Dla uproszczenia dziedziczone pola i metody zostały wyłączone z opisu poniżej.\\

Klasa \tcbox{Tile} zawiera następujące metody i pola:
\begin{lstlisting}[language=Kotlin]
@JsonIgnore
//Pole określające ile slotów ma zajmować kafelka
//Przeznaczone jest ono do ewentualnego nadpisania
//Przykładem kafelki która to robi jest kafelka thermostat
open var height = 1f

//Opcjonalny tag/nazwa kafelki
//Wyświetla się w interfejsie kafelki na ekranie dashboardu
var tag = ""

//Typ kafelki
//Pole przeznaczone do nadpisana przez klasy dziedziczące
//override var typeTag = "button"
abstract var typeTag: String

//Pola związane z kolorem i ikoną kafelki
//Tożsame pola opisane są w klasie Dashboard
abstract var iconKey: String
val iconRes: Int
var hsv: FloatArray
val pallet: Theme.ColorPallet

//Metoda uruchamiana w momencie tworzenia kafelki
//Dokładnie wtedy gdy dodawana jest ona jako nowa do dashboardu
open fun onCreateTile() {}

//Metoda uruchamiana cyklicznie dla każdej kafelki co sekundę
//Aktualizuje ona informację kiedy ostatnio kafelka odebrała wiadomość
//Informacja ta wyświetla się w interfejsie kafelki na ekranie dashboardu
//Przykład: 10 sekund temu
fun updateTimer()
\end{lstlisting}

\newpage

\subsubsection{Settings}
Prosta klasa z przeznaczeniem do przechowywania ustawień aplikacji.\\
Zawiera następujące pola:

\begin{lstlisting}[language=Kotlin]
//Obecna wersja aplikacji
var version = BuildConfig.VERSION_CODE
//Czy wysyłane powiadomienia mają być układanie w stos
var notifyStack = true
//Czy ukryć przyciski nawigacyjne
var hideNav = false
//ID ostatnio odwiedzonego dashboardu
var lastDashboardId = 0L
//Format czasu wpisów w logach dashboardów
var militaryTime = true
//Czy przechodzić od razu do ostatniego dashboardu przy starcie
var startFromLast = true
//W jakim stanie użytkownik zostawił sekcję komunikacja we właściwościach kafelek
//Przechowuje informację czy sekcja ma być rozwinięta
var mqttTabShow = true
//Czy animować kafelki przy przyjściu nowej wiadomości
var animateUpdate = true
//Czy aplikacja oczekuje na jakąś płatność opóźnioną
var pendingPurchase = false
//Czy dozwolono na pracę w tle po zamknięciu aplikacji
var fgEnabled = false
\end{lstlisting}

\subsubsection{Log}
Klasa pełniąca rolę dziennika logów. Pole jej typu posiada klasa \tcbox{Dashboard}.\\
Zawiera następujące pola i metody:

\begin{lstlisting}[language=Kotlin]
//Lista wpisów w dzienniku
val list: MutableList<LogEntry>

//Metoda dodająca nowy wpis
fun newEntry(text: String)

//Metoda czyszcząca dziennik
fun flush()
\end{lstlisting}

Wpisy w dzienniku są typu \tcbox{LogEntry}. Klasa ta dziedziczy po \tcbox{RecyclerViewItem}.
Poza dziedziczonymi metodami i polami zawiera następujące pola:

\begin{lstlisting}[language=Kotlin]
//Wartość wpisu
var text: String

//Czas dodania wpisu w formie tekstowej
val time: String
\end{lstlisting}

\newpage

\subsubsection{App}
Klasa dziedzicząca po klasie \tcbox{Application}. Jest to specjalna klasa której instancję system tworzy przed jakąkolwiek inną klasą w aplikacji. Klasa ta ma cykl życia niezależny od cyklu życia aktywności, trwający przez cały czas pracy aplikacji. Dzięki temu stan aplikacji jest niewrażliwy na wszelkie zmiany w konfiguracji które powodują ponowne uruchomienie aktywności.

\begin{lstlisting}[language=Kotlin]
class AtomApp : Application() {
    companion object {
        //Globalny stan aplikacji
        //Więcej na ten temat w kolejnym dziale
        val aps = AtomAppState()

        //Globalne odniesienie do aplikacji
        lateinit var app: AtomApp
    }

    //Metoda uruchamiana w momencie startu aplikacji
    override fun onCreate() {
        super.onCreate()
        app = this
        //Uruchomienie procesu inicjalizacji aplikacji
        Setup.initialize()
    }
}
\end{lstlisting}

\newpage

\subsubsection{AppState}
Klasa zawierająca stan aplikacji. Zawiera następujące pola i metody:

\begin{lstlisting}[language=Kotlin]
//Czy aplikacja jest licencjonowana
//Czy użytkownik wykupił wersję profesjonalną
var isLicensed = false

//Czy aplikacja przeszła już proces inicjalizacji
var isInitialized = MutableLiveData(false)

//Obiekty związane ze stanem aplikacji
var settings = Settings()
var theme = Theme()
var dashboards = mutableListOf<Dashboard>()

//Obecnie wybrane obiekty
//Dashboard i kafelka w którym obecnie jest użytkownik
//Te wartości mają między innymi związek z ekranami właściwości tych elementów
lateinit var dashboard: Dashboard
lateinit var tile: Tile
var dashboardIndex = -2

//Ścieżka do głównego folderu aplikacji
//Jest to folder dedykowany dla danej aplikacji
//Inne aplikacje nie mają do niego dostępu
//Wartość przypisywana jest w procesie inicjalizacji aplikacji
var rootFolder: String = ""

//Ścieżki do plików serializowanych obiektów
//Wartość przypisywana jest w procesie inicjalizacji aplikacji
//Więcej na temat serializacji znajduje się w dziale Storage
var path: Map<KClass<out Any>, String> = mapOf()

//Metody zmieniająca obecnie wybrany dashboard
fun setCurrentDashboard(index: Int): Boolean
fun setCurrentDashboard(id: Long): Boolean
\end{lstlisting}

\newpage

\subsubsection{Theme}
Klasa odpowiedzialna za dynamiczny motyw aplikacji wszędzie gdzie wykorzystany jest XML do tworzenia interfejsu użytkownika. Zawiera wewnętrzną klasę \tcbox{Artist} generującą monochromatyczną paletę kolorów na podstawie koloru wybranego przez użytkownika.\\

Wygenerowana paleta składa się z następujących kolorów:
\begin{itemize}[leftmargin=*]
    \item \textbf{color}: kolor bazowy
    \item \textbf{background}: kolor tła
    \item \textbf{a, b, c, d}: kolejne kolory akcentowe zbliżające się do koloru tła
\end{itemize}

\vspace{1em}

Klasa \tcbox{Artist} zawiera następujące metody i pola:

\begin{lstlisting}[language=Kotlin]
//Czy włączony jest tryb ciemny
//Zmiana wartości pola powoduje ponowne wygenerowanie palety
var isDark: Boolean

//Wybrany przez użytkownika kolor
//Zmiana wartości pola powoduje ponowne wygenerowanie palety
var hsv: FloatArray

@JsonIgnore
//Wygenerowana paleta
var pallet: ColorPallet

//Metoda generującą paletę
fun getColorPallet(
    hsv: FloatArray,
    isAltCon: Boolean = false,
    isRaw: Boolean = false
): ColorPallet
\end{lstlisting}

\newpage

Proces generowania palety przebiega w kilku krokach:

\begin{enumerate}[leftmargin=*]
    \item Generowany jest kolor tła. Nie jest on zależny od koloru źródłowego.\\ Ma on oddzielną wartość dla trybu ciemnego i dla trybu jasnego.
    \item Na podstawie koloru źródłowego generowany jest kolor bazowy. Jeżeli paleta \tcbox{isRaw} to kolor bazowy jest równy kolorowi źródłowemu.\\
    W innym przypadku jest to kolor źródłowy ograniczony w jak najmniejszym stopniu, ale na tyle aby spełniał wymóg kontrastu.
    \item Generowane są kolory akcentowe poprzez mieszanie ich z kolorem tła. Stosunek mieszania jest taki sam dla każdego koloru. Ma on oddzielną wartość dla trybu ciemnego i oddzielną wartość dla trybu jasnego:
    \begin{itemize}[leftmargin=*]
        \item \textbf{a}: kolor bazowy z kolorem tła
        \item \textbf{b}: kolor \textbf{a} z kolorem tła
        \item \textbf{c}: kolor \textbf{b} z kolorem tła
        \item \textbf{d}: kolor \textbf{c} z kolorem tła
    \end{itemize}
\end{enumerate}

\newpage

Klasa \tcbox{Theme} zawiera następujące metody i pola:
\begin{lstlisting}[language=Kotlin]
val artist = Artist()

companion object {
  //Wartość tych pól pobierana jest z obiektu artist
  val colors: ComposeColorPallet
  val isDark: Boolean
}

//Metoda mająca za zadanie ustawić motyw aplikacji dla danej grupy widoków
//Uruchamiana jest rekurencyjnie tak aby zadziałała dla każdego widoku w drzewie
private fun ViewGroup.applyTheme(p: ColorPallet) {
    for (i in 0 until this.childCount) {
        this.getChildAt(i).let {
            if (it is ViewGroup) it.applyTheme(p)
            it.defineType(p)
        }
    }

    this.defineType(p)
}

//Uruchamiaja docelową metodę odpowiedzialną za stylizację danego elementu
private fun View.defineType(p: ColorPallet) {
    when (this) {
        is RadioButton -> this.applyTheme(p)
        is Chip -> this.applyTheme(p)
        is RecyclerView -> this.applyTheme(p)
        is CircularSeekBar -> this.applyTheme(p)
        //itp.
    }
}

//Docelowe metody stylizujące
//Operują na wartości tag danego widoku aby odróżnić różne wersje stylistyczne
//Tag to dowolna wartość przypisana do widoku przez programistę
private fun FrameLayout.applyTheme(p: ColorPallet)
private fun ConstraintLayout.applyTheme(p: ColorPallet)
private fun LinearLayout.applyTheme(p: ColorPallet)
//itp.

\end{lstlisting}

\newpage

\subsection{Obiekty pomocnicze}
W Kotlinie \tcbox{object} to specjalny typ, który pozwala na tworzenie singletonów, czyli klas, które mają tylko jedną instancję. Umożliwia to łatwe tworzenie obiektów, które nie wymagają jawnego tworzenia instancji za pomocą konstruktora.\\

Ten dział jest poświęcony obiektom pomocniczym wykorzystanym w implementacji. Mają one na celu ustrukturyzowanie oraz uproszczenie pewnych funkcjonalności w kodzie.

\subsubsection{Debug}
Obiekt odpowiedzialny za centralizację wykorzystania narzędzi używanych do śledzenia błędów w aplikacji.

\begin{lstlisting}[language=Kotlin]
object Debug {
    //Dodaje dodatkowe niestandardowe informacje do raportu
    fun log(payload: String) {
        Sentry.addBreadcrumb(payload)
    }

    //Przesyła raport w przypadku wystąpienia błedu
    fun recordException(exception: Exception) {
        Sentry.captureException(exception)
    }
}
\end{lstlisting}

\subsubsection{DialogBuilder}
Obiekt ułatwiający tworzenie okien dialogowych dla interfejsu używającego XML.

\begin{lstlisting}[language=Kotlin]
object DialogBuilder {
    //Wyświetla dialog z pytaniem o potwierdzenie zadanej akcji
    inline fun Context.buildConfirm(
        message: String,
        label: String,
        textSize: Float = 20f,
        textAlign: Int = TEXT_ALIGNMENT_CENTER,
        crossinline onDeny: () -> Unit = {},
        crossinline onConfirm: () -> Unit
    )

    //Dostosowuje interfejs to wyświetlania dialogu
    //Między innymi przyciemnia to co wyświetla się za dialogiem
    fun Dialog.dialogSetup()
}
\end{lstlisting}

\newpage

\subsubsection{Pro}
Obiekt służący do zarządzania licencjonowaniem aplikacji. Licencja służy jako weryfikacja, że użytkownik wykupił wersję płatną, bez ponownego korzystania z usług Google Play. Licencjonowanie polega na utworzenie pustego pliku w folderze aplikacji. Pomimo tego, że w teorii użytkownik nie ma dostępu do tego folderu, nie jest to duże utrudnienie. Zdeterminowany użytkownik ominie również te bardziej zaawansowane. Dlatego to wprowadzone ma jedynie w prosty sposób podstawowo przeciwdziałać omijaniu zabezpieczeń, nie uprzykrzając doświadczenia pozostałym.

\begin{lstlisting}[language=Kotlin]
object Pro {
    //Sprawdza czy plik licencji istnieje
    fun getLicenceStatus(): Boolean

    //Tworzy plik licenji
    fun createLocalLicence()

    //Usuwa plik licencji
    fun removeLocalLicence()
}
\end{lstlisting}

\subsubsection{Icons}
Obiekt grupujący ikony wykorzystywane w aplikacji.\\
Posiada pole \tcbox{cats} będące listą dostępnych kategorii ikon oraz pole \tcbox{icons} będące listą obiektów typu \tcbox{Icon}.

\begin{lstlisting}[language=Kotlin]
data class Icon(
    val res: Int = 0, //ID zasobu
    val type: String = "", //Typ ikony
    val cat: String = "" //Kategoria ikony
)
\end{lstlisting}

\newpage

\subsubsection{Storage}
Obiekt odpowiedzialny za serializację obiektów. Wykorzystuje mapę \tcbox{path} zawierającą informację gdzie zapisywać obiekty danych klas.

\begin{lstlisting}[language=Kotlin]
aps.path = mapOf(
    Theme::class to "${aps.rootFolder}/theme",
    Settings::class to "${aps.rootFolder}/settings",
    Dashboard::class to "${aps.rootFolder}/dashboards"
)
\end{lstlisting}

\vspace{1em}

Jego działanie opiera się wykorzystanie uniwersalnych generycznych metod umożliwiających serializację i deserializację dowolnych obiektów lub kolekcji tych obiektów. Jedynym warunkiem jest istnienie odpowiedniego wpisu w mapie \tcbox{path}.\\

\vspace{1em}

Metody pomocnicze i mapper wykorzystywany w obiekcie.

\begin{lstlisting}[language=Kotlin]
//Mapper dostarczany przez bibliotekę Jackson
val mapper: ObjectMapper

//Metoda serializująca obiekt dowolnej klasy do String
fun Any.prepareSave(): String = mapper.writeValueAsString(this)

//Metoda zwracająca zapisany obiekt danej klasy z pliku
inline fun <reified T> getSave() = try {
    FileReader(aps.path[T::class]).readText()
} catch (_: Exception) {
    ""
}
\end{lstlisting}

\vspace{1em}

Metody służące serializacji obiektów.

\begin{lstlisting}[language=Kotlin]
//Metoda zapisująca obiekt dowolnej klasy do pliku
fun Any.saveToFile(save: String = this.prepareSave()) {
    try {
        val path = aps.path[this::class]
        File(path!!).writeText(save)
    } catch (_: Exception) {
    }
}

//Metoda zapisująca kolekcję obiektów dowolnej klasy do pliku
inline fun <reified T> Collection<T>.saveToFile(save: String = this.prepareSave()) {
    try {
        val path = aps.path[T::class]
        File(path!!).writeText(save)
    } catch (_: Exception) {
    }
}
\end{lstlisting}

\newpage

Metody służące deserializacji obiektów.

\begin{lstlisting}[language=Kotlin]
//Metoda deserializująca obiekt dowolnej klasy z pliku
inline fun <reified T> parseSave(save: String = getSave<T>()): T? =
    try {
        mapper.readValue(save, T::class.java)
    } catch (_: Exception) {
        null
    }

//Metoda deserializująca kolekcję obiektów dowolnej klasy z pliku
inline fun <reified T> parseListSave(save: String = getSave<T>()): MutableList<T> =
    try {
        mapper.readerForListOf(T::class.java).readValue(save)
    } catch (_: Exception) {
        mutableListOf()
    }
\end{lstlisting}

\vspace{1em}

W praktyce wykorzystanie powyższych metod wygląda następująco.

\begin{lstlisting}[language=Kotlin]
//Proces serializacji obiektu typu dashboard
val dashboard = Dashboard()
dashboard.saveToFile()

//Proces deserializacji obiektu typu dashboard
//Jeżeli typ pola jest określony metoda sama "domyśli się"
//jaki ma być zwrócony typ bez dodatkowych oznaczeń
dashboard = Storage.parseSave() ?: Dashboard()

//Alternatywnie
val theme = Storage.parseSave<Theme>() ?: Theme()
\end{lstlisting}

\newpage

\subsubsection{Setup}
Obiekt zawierający metody odpowiedzialne za kolejne etapy procesu inicjalizacji aplikacji.

\begin{lstlisting}[language=Kotlin]
object Setup {
    //Główna metoda rozpoczynająca inicjalizację
    fun initialize() {
        Debug.log("INIT")
        setFilesPaths()
        initializeBasicGlobals()

        CoroutineScope(Dispatchers.Default).launch {
            updateProStatus()
            checkBilling()
            checkBatteryStatus()
            configureForegroundService()
            initializeOtherGlobals()
            assignDaemons()
            finish()
        }
    }

  //Konfiguruje zmienne aps.rootFolder oraz aps.path
  private fun setFilesPaths()
  //Inicjalizuje zmienne globalne takie jak
  //aps.dashboard; aps.tile; aps.theme; aps.settings
  private fun initializeBasicGlobals()
  //Aktualizuje status licencji
  private fun updateProStatus()
  //Sprawdza czy nie ma przeprocesowanych płatności opóźnionych
  private fun checkBilling()

  //Sprawdza czy włączona jest optymalizacja baterii
  //Jeżeli tak to wyłącza pracę w tle
  private fun checkBatteryStatus()
\end{lstlisting}

\textit{Kontynuacja na drugiej stronie}

\newpage

\begin{lstlisting}[language=Kotlin]
    //Konfiguruje użycie foreground service
    //Zatrzymuje ją jeśli nie jest potrzebna
    //W innym przypadku uruchamia ją
    private suspend fun configureForegroundService() {
        //Usługa nie włączona ale działa
        if (!aps.settings.fgEnabled && service?.isStarted == true) {
            DaemonsManager.dischargeAll()
            ForegroundService.stop(app)
        }

        //Usługa jest włączona ale nie uruchomiona
        else if (aps.settings.fgEnabled && service?.isStarted != true) {
            //Zwalnia demony na wypadek jeżeli któryś pracuje
            DaemonsManager.dischargeAll()

            //Uruchamia usługę i czeka na jej start
            ForegroundService.start(app)
            ForegroundService.haltForService()
        }
    }

    //Konfiguruje pozostałe zmienne globalne
    private fun initializeOtherGlobals()

    //W zależności czy praca w tle jest włączona
    //Uruchamia demony z kontekstem aplikacji lub usługi
    private fun assignDaemons() {
        val context: Context = if (!aps.settings.fgEnabled) app
        else service!!

        DaemonsManager.assignAll(context)
    }

    private fun finish() = aps.isInitialized.postValue(true)
}
\end{lstlisting}

\newpage

\subsection{Managers}
Za wyjątkiem \tcbox{DaemonsManager} (object) są to klasy dedykowane do zarządzania określonymi zasobami oraz procesami. Ich użycie ma na celu centralizację logiki oraz zmniejszenie redundancji w kodzie.

\subsubsection{BillingManager}
Klasa służąca do procesowania zakupów oraz płatności wewnątrz aplikacji z wykorzystaniem Google Play. Do zarządzania połączeniem używa \tcbox{StatusManager}. Każda z kluczowych metod jest nieblokująca za sprawą\\
wykorzystania \tcbox{suspendCoroutine}.\\

Klasa \tcbox{BillingManager} zawiera następujące metody i pola:

\begin{lstlisting}[language=Kotlin]
val context: Context

//Czy jest jest włączony/czy pracuje
internal var isEnabled = false

//Klient dostarczany przez bibliotekę płatności
internal lateinit var client: BillingClient

private val manager = Manager()

companion object {
    //ID produktów
    var PRO = "atom_dashboard_pro"
    var DON1 = "atom_dashboard_don1"
    var DON5 = "atom_dashboard_don5"
    var DON25 = "atom_dashboard_don25"
}

fun enable() {
    isEnabled = true
    manager.dispatch(reason = "enable")
}

fun disable() {
    isEnabled = false
    manager.dispatch(reason = "disable")
}

//Buduje klienta płatności
internal fun createClient()

//Metody pomocnicze
private fun Purchase.acknowledge()
private fun Purchase.consume()

//Obsługuje potwierdzenie zakupu
//Przydziela zakupione produkty
fun onPurchased(purchase: Purchase)

//Wyświetla komunikaty związane z końcem procesu zakupu lub
//powiadamia o płatnościach opóźnionych
fun onPurchaseProcessed(purchase: Purchase)

//Pobiera informacje na temat produktów 
private suspend fun getProductDetails(id: String): MutableList<ProductDetails>?

//Pobiera listę zakupionych produktów
suspend fun getPurchases(timeout: Long = 2000): MutableList<Purchase>?

//Pobiera ceny produktów
suspend fun getPriceTags(ids: List<String>): Map<String, String>?

//Uruchamia proces zakupu
suspend fun lunchPurchaseFlow(id: String)

//Pobiera listę zakupionych produktów według filtru
//Czas wykonywania wynosi minimum wartości eta
suspend inline fun checkPurchases(
    eta: Long = 10000,
    filter: (Purchase) -> Boolean = { !it.isAcknowledged },
    onDone: (List<Purchase>?) -> Unit = {}
)

inner class Manager : StatusManager(100) {
    override fun isStable(): Boolean = when (client.connectionState) {
        CONNECTED -> isEnabled
        CONNECTING -> false
        DISCONNECTED, CLOSED -> !isEnabled
        else -> true
    }

    override fun makeStable() {
        if (client.connectionState == CLOSED) createClient()
        else if (!isEnabled) client.endConnection()
        else if (client.connectionState != CONNECTING) {
            client.startConnection(object : BillingClientStateListener {
                override fun onBillingSetupFinished(billingResult: BillingResult) {}
                override fun onBillingServiceDisconnected() {}
            })
        }
    }

    //Wstrzymuje pracę do momentu ustabilizowania połączenia
    suspend fun awaitDone(timeout: Long = 5000): Boolean = 
        withTimeoutOrNull(timeout) {
            while (!isStable()) delay(100)
            return@withTimeoutOrNull client.isReady
        } ?: false
}
\end{lstlisting}

\newpage

\subsubsection{FragmentManager}
Klasa zarządzająca nawigacją w aplikacji. Posiada następujące pola oraz metody:

\begin{lstlisting}[language=Kotlin]
//Stos fragmentów
var backstack = mutableListOf<Fragment>(MainScreenFragment())

//Obecnie wyświetlany fragment
private var currentFragment: Fragment = MainScreenFragment()

//Jeżeli zwraca prawdę to użycie systemowego przycisku/gestu cofania
//nie powoduje cofnięcia do poprzedniego fragmentu
var doOverrideOnBackPress: () -> Boolean = { false }

//Animacje tranzycji pomiędzy fragmentami
object Animations {
    val swap: (FragmentTransaction) -> Unit
    val fade: (FragmentTransaction) -> Unit
    val slideLeft: (FragmentTransaction) -> Unit
    val slideRight: (FragmentTransaction) -> Unit
    val fadeLong: (FragmentTransaction) -> Unit
}

//Zmienia obecnie wyświetlany fragment
fun replaceWith(
    fragment: Fragment,
    stack: Boolean = true,
    animation: ((FragmentTransaction) -> Unit?)? = swap
)

//Cofa do poprzedniego fragmentu
//Zwraca fałsz jeżeli stos fragmentów jest pusty
fun popBackstack(
    stack: Boolean = false,
    animation: ((FragmentTransaction) -> Unit?)? = swap
): Boolean
\end{lstlisting}

\newpage

\subsubsection{StatusManager}
Abstrakcyjna klasa mająca za zadanie periodyczne uruchamianie metody \tcbox{makeStable} od momentu wywołania metody \tcbox{dispatch} do momentu gdy metoda \tcbox{isStable} zwróci prawdę.\\

Zawiera następujące pola:
\begin{lstlisting}[language=Kotlin]
//Częstotliwość wywołania metody makeStable
private val interval: Long = if (BuildConfig.DEBUG) 1500 else 300

//Czy jest w trybie debugowania
private val debug: Boolean = false

private var job: Job? = null
var isWorking = false
\end{lstlisting}

\vspace{1em}

Metody wykonywane w kluczowych momentach:

\begin{lstlisting}[language=Kotlin]
open fun onJobStart() {}
open fun onJobDone() {}
open fun onException(e: Exception) {}
\end{lstlisting}

\vspace{1em}

Metody abstrakcyjne:

\begin{lstlisting}[language=Kotlin]
abstract fun isStable(): Boolean
abstract fun makeStable()
\end{lstlisting}

\newpage

Główna metoda \tcbox{dispatch}. Jej ponowne wywoływanie nie ma wpływu na pracę. Umożliwia opcjonalny restart obecnie uruchomionej pracy.
\begin{lstlisting}[language=Kotlin]
fun dispatch(cancel: Boolean = false, reason: String = "") {
    if (debug) Debug.log("SM_DISPATCH [$reason]")

    if (cancel) {
        if (debug) Debug.log("SM_CANCEL_JOB")
        job?.cancel()
    }

    if (job != null && job?.isActive == true) return

        job = GlobalScope.launch(Dispatchers.IO) {
        if (debug) Debug.log("SM_JOB_LAUNCH")
        try {
            //Pierwsza iteracja
            if (!isStable()) {
                isWorking = true
                onJobStart()
                makeStable()
                delay(interval)
            }

            //Pozostałe iteracje
            while (!isStable()) {
                makeStable()
                delay(interval)
            }

            isWorking = false
            onJobDone()
            if (debug) Debug.log("SM_SETTLE")
        } catch (e: Exception) {
            onException(e)
            delay(interval)
            dispatch(true)
        }
    }
}
\end{lstlisting}

\newpage

\subsubsection{DaemonsManager}
Obiekt odpowiedzialny za przydzielanie oraz zwalnianie demonów. Posiada następujące metody:
\begin{lstlisting}[language=Kotlin]
fun assignAll(context: Context) = aps.dashboards.forEach { assign(it, context) }

fun dischargeAll() {
    Debug.log("DM_DISCHARGE_ALL")
    aps.dashboards.forEach { discharge(it) }
}

fun assign(dashboard: Dashboard, context: Context) = try {
    Debug.log("DM_ASSIGN")

    dashboard.apply {
        daemon = Daemon(context, this, type)
        daemon?.notifyAssigned()
    }
} catch (e: Exception) {
    Debug.recordException(e)
}

fun discharge(dashboard: Dashboard) = try {
    Debug.log("DM_DISCHARGE")
    dashboard.daemon?.notifyDischarged()
} catch (e: Exception) {
    Debug.recordException(e)
}
\end{lstlisting}

\subsubsection{ToolbarManager}
Klasa definiująca działanie paska narzędzi dostępnego z poziomu interfejsu dashboardu. Pełni rolę warstwy abstrakcji, tak aby uprościć implementację fragmentu odpowiedzialnego za tą część interfejsu użytkownika.

\newpage

\subsection{Interfejs użytkownika}
\subsubsection{XML}
Znacząca część interfejsu użytkownika została zaimplementowana z wykorzystaniem Jetpack Compose. Nie całość, ponieważ niestety pewne funkcjonalności ekranów używających \tcbox{RecyclerView} byłyby trudne do odtworzenia wykorzystując tą technologię zamiast XML.\\

\tcbox{RecyclerView} jest to narzędzie dostępne jedynie z XML. Umożliwia ono tworzenie dynamicznych list które procesują jedynie elementy obecnie wyświetlane na ekranie.\\

Posiada ono również stosunkowo łatwą do implementacji możliwość obsługi gestu zmiany kolejności wyświetlanych elementów (ang. drag and drop) wraz z towarzyszącym temu animacjami.\\

Samodzielna implementacja takiej funkcjonalności przy listach jednowymiarowych nie stanowi problemu. Jednak sprawa znacząco się komplikuje w przypadku list dwuwymiarowych, gdzie elementy mają różne rozmiary, tak jak to jest w tym przypadku.\\

Z tego powodu, ekran główny wyświetlający listę dashboardów oraz ekran wyświetlający sam dashboard z dwuwymiarową listą kafelek wykorzystują XML, a co za tym idzie również interfejsy samych kafelek oraz dialogi wyświetlane na tych ekranach.

\newpage

\subsubsection{Fragmenty}
Za każdy ekran aplikacji jest odpowiedzialny odrębny fragment:

\tabeladwiekolumny{0.4}{0.6} {
    \tabeladwiekolumnywiersz
    {\textbf{DashboardFragment}}
    {ekran dashboardu}

    %\tabeladwiekolumnywiersz
    %{\textbf{DashboardNewFragment}}
    %{ekran dodawania nowego dashboardu}

    \tabeladwiekolumnywiersz
    {\textbf{DashboardPropertiesFragment}}
    {ekran właściwości dashboardu}

    \tabeladwiekolumnywiersz
    {\textbf{LoadingFragment}}
    {ekran ładowania}

    \tabeladwiekolumnywiersz
    {\textbf{MainScreenFragment}}
    {ekran główny z listą dashboardów}

    \tabeladwiekolumnywiersz
    {\textbf{SettingsFragment}}
    {ekran ustawień}

    \tabeladwiekolumnywiersz
    {\textbf{ThemeFragment}}
    {ekran konfiguracji motywu}

    \tabeladwiekolumnywiersz
    {\textbf{TileIconFragment}}
    {ekran konfiguracji ikony kafelki}

    \tabeladwiekolumnywiersz
    {\textbf{TileNewFragment}}
    {ekran dodawania nowej kafelki}

    \tabeladwiekolumnywiersz
    {\textbf{TilePropertiesFragment}}
    {ekran konfiguracji kafel}
}

\subsubsection{RecyclerView}
Jest to narzędzie dostępne jedynie z XML. Umożliwia ono tworzenie dynamicznych list które procesują jedynie elementy obecnie wyświetlane na ekranie. Posiada ono również stosunkowo łatwą do implementacji możliwość obsługi gestu zmiany kolejności wyświetlanych elementów (ang. drag and drop) wraz z towarzyszącym temu animacjami. W przedstawianej aplikacji \tcbox{RecyclerView} jest wykorzystywany między innymi do wyświetlania listy dashboardów na ekranie głównym oraz do wyświetlania siatki kafelek na ekranie dashboardu. W skład implementacji \tcbox{RecyclerView} wchodzą trzy klasy opisane w kolejnych rozdziałach.

\newpage

\subsubsection{RecyclerViewAdapter}
Generyczna klasa dziedzicząca po \tcbox{ListAdapter}. Rolą adaptera jest zarządzanie wyświetlaną listą.\\

Poza metodami dziedziczonymi \tcbox{RecyclerViewAdapter} został rozszerzony o:\\

\textbf{1. Użytkowe funkcje lambda wywoływane w kluczowych momentach}

\begin{lstlisting}[language=Kotlin]
var onBindViewHolder: (item, ViewHolder, Int) -> Unit = { _, _, _ -> }
var onItemClick: (item) -> Unit = {}
var onItemLongClick: (item) -> Unit = {}
var onItemRemoved: (item) -> Unit = {}
var onItemMarkedRemove: (Int, Boolean) -> Unit = { _, _ -> }
var onItemEdit: (item) -> Unit = {}
\end{lstlisting}

\vspace{1em}

\textbf{2. Metody pomocnicze}

\begin{lstlisting}[language=Kotlin]
//Oznacza element jako ten do usunięcia
//Zmienia go wizualnie sygnalizując to użytkownikowi
fun markItemRemove(position: Int)

//Usuwa oznaczone przez użytkownika elementy listy
fun removeMarkedItems()

//Usuwa element z określonej pozycji listy
fun removeItemAt(pos: Int, notify: Boolean = true)
\end{lstlisting}

\vspace{1em}

\textbf{3. Rozszerzoną obsługę trybu edycji}

\begin{lstlisting}[language=Kotlin]
var editMode = Modes()
open inner class Modes {
    //Obecny tryb
    private var mode = -1

    //Funkcja lambda wywoływana w momencie zmiany trybu
    var onSet: (Modes) -> Unit = {}

    //Czy znajduje się w danym trybie
    val isNone: Boolean
    val isSwap: Boolean
    val isRemove: Boolean
    val isEdit: Boolean

    //Metody ustawiające określony tryb edycji
    fun setNone()
    fun setSwap()
    fun setRemove()
    fun setEdit()
}
\end{lstlisting}

\newpage

\subsubsection{RecyclerViewItem}
Abstrakcyjna klasa po której dziedziczy każda klasa która jednocześnie reprezentuje elementy danej listy w aplikacji. Jedną z takich klas jest klasa \tcbox{Tile}.\\

W skład tej klasy wchodzą następujące metody i pola:
\begin{lstlisting}[language=Kotlin]
//Unikatowy identyfikator elementu
var id: Long

//Identyfikator zasobu XML będącego interfejsem graficznym elementu
abstract val layout: Int

//Odniesienie do ViewHolder danego elementu
@JsonIgnore
var holder: RecyclerViewAdapter.ViewHolder? = null

//Adapter do którego jest przypisany element
@JsonIgnore
var adapter: RecyclerViewAdapter<*>? = null

//Metody związane z RecyclerView

open fun onCreateViewHolder(
    parent: ViewGroup,
    viewType: Int
): RecyclerViewAdapter.ViewHolder

open fun onViewAttachedToWindow(holder: RecyclerViewAdapter.ViewHolder)

open fun onBindViewHolder(
    holder: RecyclerViewAdapter.ViewHolder,
    position: Int
)

//Metoda odpowiedzialna za stylizację elementu
open fun onSetTheme(holder: RecyclerViewAdapter.ViewHolder)

//Metody reagujące na zdarzenia związane z elementem
open fun onTouch(v: View, e: MotionEvent)
open fun onClick(v: View, e: MotionEvent)
open fun onEdit(isEdit: Boolean)
\end{lstlisting}

\newpage

Dodatkowo klasa \tcbox{RecyclerViewItem} posiada system flag określający w jakim stanie znajduje się element.

\begin{lstlisting}[language=Kotlin]
@JsonIgnore
var flag = Flags()

inner class Flags {
    //Obecna flaga
    private var flag = -1

    //Czy dana flaga jest ustawiona
    val isNone
    val isRemove

    //Ustawia daną flagę
    fun setNone() = setFlag(-1)
    fun setRemove() = setFlag(1)

    //Aktualizuje wizualnie element w zależności od flagi
    private fun show()
}
\end{lstlisting}

\subsubsection{ItemTouchCallback}
Ostatnia klasa wchodząca w skład implementacji \tcbox{RecyclerView}. Odpowiada ona za gesty, animacje oraz zmiany kolejności elementów listy.

\subsubsection{Wykrywanie gestów}
Wykrywanie globalnych gestów w aplikacji opiera się o własną implementację \tcbox{ConstraintLayout} oraz \tcbox{LinearLayout}, tak aby uzyskać dostęp do metody \tcbox{onInterceptTouchEvent} umożliwiającej przechwycenie zdarzeń związanych z dotykiem. Wykorzystanie tych klas jako \tcbox{root layout} interfejsu użytkownika pozwoliło na implementację dowolnych gestów w aplikacji.