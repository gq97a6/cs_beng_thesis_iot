\section{Faza analizy i planowania}

\subsection{Koncepcja}
Aplikacja mobilna która działała jako uniwersalny interfejs sterowania różnymi urządzeniami IoT. Projekt zakłada użycie MQTT jako protokołu komunikacyjnego z uwagi na szerokie zastosowanie w domenie IoT ale również przewiduje możliwość przyszłego rozszerzenia o inne takie pozwalające na komunikację bezpośrednio z urządzeniem lub siecią urządzeń. (artykuły HiveMQ - \cite{mqttincreasedadoption}\cite{mqttadoptionishigh}) Użytkownik korzystając z szerokiej gamy predefinowanych kafelek może stworzyć interfejs pozwalający sterować dowolnym urządzeniem o ile implementuje ono jeden z obsługiwanych protokołów. Przedstawiana aplikacja może stać się centrum sterowania inteligentnego domu gdzie nadzoruje pracę całej automatyki lub też zostać wykorzystana jako interfejs użytkownika w przypadku projektów samodzielnych urządzeń lub prototypów gdzie tworzenie dedykowanego jest zbędnym wydatkiem. Z racji na nacisk na uniwersalne protokoły cała komunikacja może pozostać pod kontrolą użytkownika wykorzystując całkowicie jego infrastrukturę.

\subsection{Konkurencyjne rozwiazania}

\textbf{MQTT Dash} stworzone przez \textbf{Routix software} oraz \textbf{IoT MQTT Panel} autorstwa \textbf{Rahul Kundu}: obie te aplikacje posiadają przestarzały interfejs użytkownika co dyskwalifikuje je na start. Dodatkowym minusem jest fakt że nie umożliwiają użycia własnych certyfikatów SSL czy certyfikatów klienta. Pierwsza z wymienionych nie otrzymała aktualizacji od 2017 roku co sugeruje że jest porzuconym projektem.\\

\textbf{Mqtt Dashboard} stworzone przez \textbf{Vetru} jest jedynym bezpośrednim konkurentem. Obsługuje zarówno własne certyfikaty SSL oraz certyfikaty klientów. Posiada aktualny i przejrzysty interfejs który może przypaść wielu do gustu. Ostatnia aktualizacja nastąpiła w 2021 roku.

\subsection{Profil użytkownika docelowego}
Użytkownik zaznajomiony z technologią posiadający wiedzę oraz umiejętności techniczne związane ze światem IoT. Wymagający szerokich możliwości konfiguracji i dostosowywania pod własne potrzeby. Zwracający uwagę na prywatność oraz ceniący sobie używanie własnych rozwiązań ponad te z zamkniętego źródła.

\subsubsection{Model biznesowy}
Aplikacja dostępna w darmowej wersji posiadającej wszystkie funkcjonalności. Możliwość zakupu wersji rozszerzonej zdejmującej ograniczenie co do ilości stworzonych dashboardów.

\newpage

\subsection{Przykłady użycia}
W tej sekcji znajdują się trzy przykładowe scenariusze wykorzystania prezentowanej aplikacji. W opisach celowo pominięto serwer MQTT.

\subsubsection{Automatyka domowa}
Przykład użycia aplikacji do sterowania automatyką domową:\\

\textbf{Sterownik oświetlenia RGB} - jego rolą jest kontrolowanie oświetlenia. Nie wysyła żadnych nowych informacji więc jedynie subskrybuje cztery \tcbox{topic} związane z ustawionym kolorem, stanem, trybem oraz jasnością oświetlenia.\\

\textbf{Sterownik z przekaźnikiem} - jego odpowiedzialny za sterowanie zasilaniem podłączonego urządzenia, w tym przypadku wentylatora. Subskrybuje jeden topic. W zależności od payload włącza lub wyłącza zasilanie. W alternatywnej implementacji mógłby jedynie przełączać stan urządzenia nie zważając na payload.\\

\textbf{Sterownik ogrzewania podłogowego} - ma za zadanie kontrolować ogrzewaniem w pokoju. Z tego powodu subskrybuje dwa \tcbox{topic} zawierające zadaną oraz aktualną temperaturę.\\

\textbf{Czujnik klimatu} - przesyła na dedykowane \tcbox{topic} temperaturę i wilgotność powietrza oraz ciśnienie atmosferyczne.\\

\textbf{Aplikacja} - wyświetla interfejs użytkownika pokazujący informacje publikowane przez czujniki jak i dający możliwość kontroli pracy pozostałych urządzeń. Z tego powodu subskrybuje i publikuje do wszystkich wyżej opisanych \tcbox{topic}.\\

\newpage

\subsubsection{Rolnictwo precyzyjne}
Przykład użycia aplikacji do rolnictwa precyzyjnego:\\

\textbf{Czujniki wilgotności gleby} - mierzą poziom wilgotności w glebie na różnych głębokościach.\\

\textbf{Czujniki upraw} - mierzące temperaturę i wilgotność powietrza w polu.\\

\textbf{Czujnik nasłonecznienia} - mierzy ilość światła słonecznego docierającego do roślin.\\

\textbf{System nawadniania} - sterowane zdalnie na podstawie danych z sensorów.\\

\textbf{Stacja meteorologiczna} - mierzy różne parametry meteorologiczne, takie jak prędkość i kierunek wiatru, opady, ciśnienie atmosferyczne.\\

\textbf{Serwer} - na podstawie zebranych pomiarów steruje automatycznym nawadnianiem. Powiadamia o wszelkich anomaliach. Agreguje oraz archiwizuje zebrane dane do przyszłej długoterminowej analizy.\\

\textbf{Aplikacja} - wyświetla interfejs użytkownika pokazujący informacje publikowane przez czujniki jak i dający możliwość kontroli pracy pozostałych urządzeń. Z tego powodu subskrybuje i publikuje do wszystkich wyżej wymienionych \tcbox{topic}.\\

\newpage

\subsubsection{Inteligentne budynki}
Przykład użycia aplikacji w kontekście inteligentnych budynków:\\

\textbf{Czujniki obecności} - wykrywają obecność osób w pomieszczeniach, co pozwala na optymalizację oświetlenia i klimatyzacji.\\

\textbf{Inteligentne termostaty} - kontrolują temperaturę w pomieszczeniach na podstawie danych z czujników obecności oraz preferencji użytkowników. Subskrybują \tcbox{topic} związany z ustawioną temperaturą oraz publikują aktualną temperaturę.\\

\textbf{Sterowniki oświetlenia} - zarządzają oświetleniem w budynku, dostosowując jego intensywność i barwę w zależności od pory dnia, obecności osób i naturalnego światła. Subskrybują \tcbox{topic} związany z ustawieniem oświetlenia.\\

\textbf{Czujniki jakości powietrza} - monitorują poziom CO2, wilgotność oraz inne parametry jakości powietrza, co pozwala na optymalizację wentylacji i klimatyzacji. Publikują dane na dedykowany topic.\\

\textbf{System zarządzania energią} - monitoruje zużycie energii w budynku, optymalizując jej wykorzystanie poprzez zarządzanie oświetleniem, ogrzewaniem, wentylacją i klimatyzacją. Publikuje i subskrybuje dane dotyczące zużycia energii.\\

\textbf{Serwer budynkowy} - agreguje dane ze wszystkich czujników i urządzeń, analizuje je oraz podejmuje decyzje dotyczące zarządzania budynkiem. Powiadamia odpowiednie służby o wszelkich anomaliach i awariach. Archiwizuje dane do przyszłej analizy i planowania.\\

\textbf{Aplikacja} - wyświetla interfejs użytkownika pokazujący informacje publikowane przez czujniki oraz daje możliwość kontroli pracy pozostałych urządzeń. Z tego powodu subskrybuje i publikuje do wszystkich wyżej wymienionych \tcbox{topic}.\\