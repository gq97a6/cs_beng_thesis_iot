\section{Słownik pojęć}
\begin{enumerate}[leftmargin=*]
\item \textbf{Android} - \textit{system operacyjny stworzony przez Google, oparty na jądrze Linux, przeznaczony głównie dla urządzeń mobilnych, takich jak smartfony i tablety.}
\item \textbf{API} - \textit{(ang. Application Programming Interface) interfejs programistyczny aplikacji, który umożliwia komunikację między różnymi programami lub komponentami oprogramowania.}
\item \textbf{App Store} - \textit{platforma dystrybucji aplikacji dla urządzeń z systemem iOS, zarządzana przez firmę Apple.}
\item \textbf{Asynchroniczność} - \textit{sposób wykonywania operacji w programowaniu, w którym zadania są realizowane niezależnie od siebie, bez blokowania głównego wątku aplikacji.}
\item \textbf{Dashboard} - \textit{interfejs użytkownika zawierający podsumowanie kluczowych informacji i wskaźników, często w formie wizualnej, takich jak wykresy czy tabele.}
\item \textbf{Firebase} - \textit{platforma stworzona przez Google, oferująca szeroki zestaw narzędzi i usług wspierających tworzenie, rozwój oraz utrzymanie aplikacji mobilnych i webowych. Obejmuje m.in. bazę danych w czasie rzeczywistym, uwierzytelnianie użytkowników, powiadomienia push, hosting, analitykę aplikacji, a także narzędzia takie jak Crashlytics do monitorowania błędów i awarii aplikacji w celu poprawy jej stabilności i jakości.}
\item \textbf{Fragment} - \textit{moduł interfejsu użytkownika w Androidzie, który może być wielokrotnie używany w różnych aktywnościach lub aplikacjach.}
\item \textbf{Git} - \textit{system kontroli wersji umożliwiający śledzenie zmian w kodzie źródłowym i współpracę wielu programistów nad jednym projektem.}
\item \textbf{Google Play} - \textit{oficjalny sklep z aplikacjami dla urządzeń z systemem Android, zarządzany przez Google.}
\item \textbf{Google Play Console} - \textit{platforma stworzona przez Google dla deweloperów aplikacji na Androida, umożliwiająca zarządzanie publikacją aplikacji w sklepie Google Play. Oferuje narzędzia do monitorowania statystyk pobrań, zarządzania opiniami użytkowników, testowania aplikacji, analizy wydajności, monitorowania błędów (Crash Reports) oraz konfiguracji funkcji takich jak subskrypcje, zakupy w aplikacji czy kampanie promocyjne.}
\item \textbf{IoT} - \textit{(ang. Internet of Things) koncepcja sieci urządzeń połączonych z Internetem, które mogą wymieniać dane i działać autonomicznie, np. inteligentne domy czy urządzenia przemysłowe.}
\item \textbf{Jetpack Compose} - \textit{nowoczesny framework UI dla Androida, oparty na deklaratywnym podejściu do tworzenia interfejsów użytkownika.}
\item \textbf{JSON} - \textit{(ang. JavaScript Object Notation) lekki format wymiany danych, oparty na strukturze klucz-wartość, łatwy do odczytu zarówno dla ludzi, jak i maszyn.}
\item \textbf{Kotlin} - \textit{nowoczesny język programowania stworzony przez JetBrains, oficjalnie wspierany przez Google jako główny język do tworzenia aplikacji na Androida.}
\item \textbf{Kotlin Coroutines} - \textit{mechanizm w języku Kotlin umożliwiający łatwe tworzenie i zarządzanie operacjami asynchronicznymi w sposób sekwencyjny.}
\item \textbf{Material Design} - \textit{język projektowania stworzony przez Google, który definiuje zasady tworzenia estetycznych, spójnych i intuicyjnych interfejsów użytkownika.}
\item \textbf{Programowanie zdarzeniowe} - \textit{paradygmat programowania, w którym przepływ programu jest sterowany przez zdarzenia, takie jak kliknięcia użytkownika, dane z czujników czy wiadomości sieciowe.}
\item \textbf{Rolnictwo precyzyjne} - \textit{nowoczesna metoda zarządzania gospodarstwami rolnymi, wykorzystująca technologie, takie jak IoT, GPS czy drony, w celu optymalizacji produkcji i minimalizacji strat.}
\item \textbf{Sentry} - \textit{narzędzie do monitorowania błędów i zarządzania nimi w aplikacjach, które pozwala na szybkie diagnozowanie problemów w kodzie.}
\item \textbf{Serializacja} - \textit{proces przekształcania obiektu w formę, która może być przechowywana lub przesyłana, np. do formatu JSON lub XML.}
\item \textbf{Single-Activity} - \textit{podejście w programowaniu aplikacji na Androida, w którym cała aplikacja składa się z jednej aktywności, a jej zawartość jest zarządzana za pomocą fragmentów.}
\item \textbf{XML} - \textit{(ang. eXtensible Markup Language) język znaczników używany do przechowywania i przesyłania danych w strukturze hierarchicznej, często wykorzystywany w konfiguracji aplikacji i interfejsach użytkownika.}
\end{enumerate}