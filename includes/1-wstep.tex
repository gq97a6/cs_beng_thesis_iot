\section{Wstęp}
Współczesny świat charakteryzuje się szybkim rozwojem technologii Internetu Rzeczy. Prowadzi to do co raz większej roli automatyzacji w naszym codziennym życiu. Urządzenia te, od inteligentnych żarówek, przez systemy ogrzewania, po zaawansowane systemy bezpieczeństwa, mają na celu ułatwienie codziennych czynności i poprawę jakości życia użytkowników. Jednakże, rosnąca liczba i różnorodność tych urządzeń rodzi nowe wyzwania, zwłaszcza w kontekście ich efektywnego zarządzania i integracji.\\

Jednym z głównych problemów jest fakt, że producenci urządzeń IoT często stosują zamknięte protokoły komunikacyjne, co utrudnia lub nawet uniemożliwia ich integrację w ramach jednego, spójnego systemu. Taka sytuacja zmusza użytkowników do korzystania z wielu różnych aplikacji do zarządzania swoimi urządzeniami, co jest niewygodne i ogranicza potencjał automatyzacji.\\

Ważnym punktem również wymagającym poruszenia jest wątpliwe bezpieczeństwo jak i brak przejrzystości przy wykorzystaniu zamknietych rozwiązań jakie oferują producenci urządzeń IoT przeznaczonych na rynek konsumencki. Konsument rzadko ma możliwość dowiedzieć się co się dzieje z jego danymi lub kto ma dostęp do jego urządzeń. Infrastruktury producentów często wykorzystują wiele serwerów rozsianych po całym świecie nad którymi użytkownik nie ma kontroli.\\

Celem niniejszej pracy inżynierskiej jest zaprojektowanie i implementacja aplikacji mobilnej, która działałaby jako uniwersalny panel do sterowania różnorodnymi urządzeniami IoT. Aplikacja ta ma na celu zaoferowanie użytkownikom centralnego punktu dostępu do wszystkich ich inteligentnych urządzeń.\\

Główna część pracy została podzielona na poszczególne fazy:\\

1. Faza analizy i planowania gdzie analizowana jest cała koncepcja wraz z rynkiem i profilem użytkownika docelowego.\\

2. Faza projektowania gdzie stawiane są wymagania jakie spełnić ma aplikacja i omawiane są wykorzystane technologie oraz metodyka pracy.\\

3. Faza implementacji gdzie opisane są poszczególne etapy tworzenia aplikacji.\\

4. Faza testów końcowych i publikacji gdzie pokazane są ostateczne testy gotowego produktu jaki i publikacja aplikacji na platformie Google Play.\\

5. Faza utrzymania gdzie opisane są napotkane trudności oraz plany na przyszły rozwój aplikacji.\\