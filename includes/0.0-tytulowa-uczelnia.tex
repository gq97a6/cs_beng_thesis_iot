\thispagestyle{empty}

\vspace*{2em}

\begin{center}
    \includegraphics{pjatk}
\end{center}

\vspace{3em}

\begin{center}
    \large
    \textbf{Wydział Informatyki}\\
    Filia w Gdańsku\\
    
    \vspace{3em}

    \textbf{Szymon Kogut}\\
    Numer albumu: 24271\\
    Nazwa specjalizacji: Cyberbezpieczeństwo\\
    
    \vspace{3em}

    \begin{center}
        \huge
        \textbf{Uniwersalny panel sterowania IoT}
        \Large
        Universal IoT control panel
    \end{center}

    \vspace{3em}
    
    \hfill
    \begin{varwidth}{\linewidth}
        \raggedright
        \textbf{Rodzaj pracy}\\
        Inżynierska\\

        \vspace{1em}

        \textbf{Imię i nazwisko promotora}\\
        dr Tadeusz Puźniakowski
    \end{varwidth}
\end{center}

\vfill

\begin{center}
    Gdańsk \today
\end{center}

\newpage
{
    \large
    \textbf{Streszczenie:}
    Praca dyplomowa dotyczy projektowania i implementacji aplikacji mobilnej która funkcjonuje jako uniwersalny interfejs sterowania dla urządzeń IoT. Celem aplikacji jest umożliwienie użytkownikom zdalnie zarządzać różnorodnymi inteligentnymi urządzeniami za pośrednictwem jednej platformy z wykorzystaniem protokołu MQTT. W pracy omówiono wybrane technologie programistyczne które zostały użyte do stworzenia aplikacji oraz wyzwania związane z zapewnieniem bezpieczeństwa i prywatności w ramach rozproszonego środowiska IoT. Aplikacja jest dostępna na urządzenia z systemem Android. Do jej stworzenia wykorzystano język Kotlin oraz bibliotekę Jetpack Compose.\\

    \textbf{Słowa kluczowe:} iot, mqtt, android, aplikacja, kotlin
}