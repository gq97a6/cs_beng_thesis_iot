\thispagestyle{empty}

\vspace*{2em}

\begin{center}
    \includegraphics{pjatk}
\end{center}

\vspace{2em}

\begin{center}
    \huge
    \textbf{Uniwersalny panel sterowania IoT}
    \Large
    Universal IoT control panel
\end{center}

\vspace{4em}

\textbf{Autor:} Szymon Kogut\\
\textbf{Numer albumu:} 24271\\

\textbf{Tryb studiów:} Stacjonarne\\
\textbf{Kierunek studiów:} Informatyka\\
\textbf{Specjalizacja:} Cyberbezpieczeństwo\\

\textbf{Rodzaj pracy:} Inżynierska\\
\textbf{Promotor:} dr Tadeusz Puźniakowski\\

\textbf{Streszczenie:}
Praca dyplomowa dotyczy projektowania i implementacji aplikacji mobilnej która funkcjonuje jako uniwersalny interfejs sterowania dla urządzeń IoT. Celem aplikacji jest umożliwienie użytkownikom zdalnie zarządzać różnorodnymi inteligentnymi urządzeniami za pośrednictwem jednej platformy z wykorzystaniem protokołu MQTT. W pracy omówiono wybrane technologie programistyczne które zostały użyte do stworzenia aplikacji oraz wyzwania związane z zapewnieniem bezpieczeństwa i prywatności w ramach rozproszonego środowiska IoT. Aplikacja jest dostępna na urządzenia z systemem Android. Do jej stworzenia wykorzystano język Kotlin oraz bibliotekę Jetpack Compose.\\

\textbf{Słowa kluczowe:} iot, mqtt, android, aplikacja, kotlin

\vfill

\begin{center}
    Gdańsk \today
\end{center}