\thispagestyle{empty}

\vspace*{2em}

\begin{center}
    \includegraphics{pjatk}
\end{center}

\vspace{2em}

\textbf{Kierunek studiów:} Informatyka
\hfill
\textbf{Rodzaj studiów:} Stacjonarne

\vspace{2em}

\begin{center}
    \Huge
    Praca dyplomowa
\end{center}

\vspace{3em}

\textbf{Temat pracy:} Uniwersalny panel sterowania IoT\\
\textbf{Temat w języku angielskim:} Universal IoT control panel\\
\textbf{Promotor:} dr. Tadeusz Puźniakowski\\

\textbf{Autor:} Szymon Kogut\\
\textbf{Numer albumu:} s24271\\

\textbf{Streszczenie:}
\textsl{Praca dyplomowa dotyczy projektowania i implementacji aplikacji mobilnej która funkcjonuje jako uniwersalny interfejs sterowania dla urządzeń IoT. Celem aplikacji jest umożliwienie użytkownikom zdalnie zarządzać różnorodnymi inteligentnymi urządzeniami za pośrednictwem jednej platformy z wykorzystaniem protokołu MQTT. W pracy omówiono wybrane technologie programistyczne które zostały użyte do stworzenia aplikacji oraz wyzwania związane z zapewnieniem bezpieczeństwa i prywatności w ramach rozproszonego środowiska IoT. Aplikacja jest dostępna na urządzenia z systemem Android. Do jej stworzenia wykorzystano język Kotlin oraz bibliotekę Jetpack Compose.}\\

\textbf{Słowa kluczowe:} iot, mqtt, android, aplikacja, kotlin

\vfill

\begin{center}
    Gdańsk \today
\end{center}