\section{Faza testów końcowych i publikacji}

\subsection{Testy końcowe}
\subsubsection{Testy manualne}
Testy manualne polegały na długoterminowym korzystaniu z aplikacji w różnych scenariuszach użytkowania oraz weryfikacji czy spełniono wszystkie postawione wymagania. Skupiono się na następujących aspektach:\\

\hspace{3em}

\textbf{Wymagania:} Projekt spełnia wszystkie postawione wymagania z priorytetem wyższym bądź równym \textbf{SHOULD}.\\

\hspace{3em}

\textbf{Interfejs użytkownika:} Ocena intuicyjności i responsywności interfejsu, w tym łatwości nawigacji oraz dostępności kluczowych funkcji.\\

Scenariusze obejmowały:

\begin{itemize}[leftmargin=*]
    \item Sprawdzanie czy użytkownik może łatwo znaleźć potrzebne informacje.
    \item Testowanie reakcji aplikacji na różne rozdzielczości i orientacje ekranu.
\end{itemize}

\hspace{3em}

\textbf{Stabilność:} Obserwacja zachowania aplikacji w dłuższych sesjach użytkowania, w tym sprawdzanie, czy aplikacja nie ulega awariom ani nie zawiesza się.\\

Scenariusze obejmowały:

\begin{itemize}[leftmargin=*]
    \item Śledzenie stanu aplikacji długo pracującej w tle.
    \item Dodawanie wielu jednocześnie pracujących dashboardów.
    \item Testowanie aplikacji w warunkach słabego połączenia internetowego.
    \item Obserwowanie jak aplikacja radzi sobie z utratą połączenia z internetem.
    \item Sprawdzanie czy aplikacja utrzymuje połączenie z brokerem.
    \item Testowanie czy aplikacja utrzymuje stałe połączenie z brokerem.
    \item Sprawdzanie czy aplikacja wznawia połączenie z brokerem po jego utracie.
\end{itemize}

\hspace{3em}

\textbf{Wydajność:} Monitorowanie czasu ładowania danych oraz reakcji na interakcje użytkownika, co miało na celu ocenę płynności działania aplikacji.\\

Scenariusze obejmowały:

\begin{itemize}[leftmargin=*]
    \item Pomiar czasu ładowania aplikacji.
    \item Testowanie szybkości reakcji aplikacji na różne interakcje użytkownika, takie jak przewijanie, klikanie przycisków, czy wprowadzanie danych.
    \item Analiza zużycia zasobów systemowych podczas normalnego użytkowania.
    \item Sprawdzanie płynności list przy dużej ilości wyświetlanych elementów.
\end{itemize}

\newpage

\subsubsection{Testy z różnymi konfiguracjami serwera MQTT}
W celu oceny kompatybilności aplikacji z różnymi konfiguracjami serwera MQTT, przeprowadzono testy z użyciem różnych protokołów oraz ustawień:\\

\textbf{Obsługa różnych protokołów:} Testowano aplikację z serwerami obsługującymi protokoły MQTT 3.1, 3.1.1 oraz 5.0, aby zweryfikować, czy aplikacja prawidłowo interpretuje i przetwarza wiadomości w każdym z tych standardów.\\

\textbf{Obsługa różnych form uwierzytelniania:} Testowano czy aplikacja poprawnie obsługuje wszystkie warianty oraz kombinacje uwierzytelniania: za pośrednictwem certyfikatów klienta, a także z wykorzystaniem hasła i loginu.\\

\textbf{Obsługa certyfikatów self-signed:} Testowano czy aplikacja poprawnie łączy się z serwerami skonfigurowanymi przy użyciu certyfikatów self-signed.

\newpage

\subsection{Publikacja na platformie Google Play}
Aplikacja jest dostępna do pobrania na platformie Google Play\\pod nazwą \textbf{Atom dashboard - MQTT and IoT}.
Na dzień 01.12.2024 aplikacja posiada \textbf{358} unikalnych użytkowników.\\

Poniżej znajduje się lista najpopularniejszych krajów:

\tabeladwiekolumny{0.30}{0.70}{
    \tabeladwiekolumnywykres{Indonezja(25)}{25}{31}
    \tabeladwiekolumnywykres{Rosja(31)}{31}{31}
    \tabeladwiekolumnywykres{Niemcy(27)}{27}{31}
    \tabeladwiekolumnywykres{Brazylia(19)}{19}{31}
    \tabeladwiekolumnywykres{Stany Zjednoczone(20)}{20}{31}
    \tabeladwiekolumnywykres{Indie(16)}{16}{31}
    \tabeladwiekolumnywykres{Korea Południowa(13)}{13}{31}
    \tabeladwiekolumnywykres{Wielka Brytania(11)}{11}{31}
    \tabeladwiekolumnywykres{Malezja(13)}{13}{31}
}